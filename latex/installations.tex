
\section{Softwares installation}

Décrire votre travail (environ 6 pages)
\begin{itemize}
	\item architecture de votre solution (vision haut niveau)
	\item implémentation de votre solution (détail technique) ou de la partie la plus intéressante si la place manque

\end{itemize}

In this part I will discuss the diffenrents softwares I have installed for the company. Firstofall I will focus on the research I have done. Then the details of the installation will be detailed. 
\subsection{Researches}
%\includegraphics[scale=0.55]{Images/redmine.png} \\

\includegraphics[width=\textwidth]{Images/redmine.png}
\newline
\\
I dind't know a lot about project managing so I had to look what was Redmine for and what this application will allow us to do. \\ 
Redmine is a powerful web based project managing which allow bugs tracking. It is possible to add a lot of functionalities to this application in intalling  some plugings into it. \\ 
\newpage
I read about all the plugings for Redmine and finally chose some of them  
	\begin{figure}[h]
		\includegraphics[scale=0.1]{Images/jenkins.jpeg} 
		Jenkins. \\
		This application provides an easy to use continuous integration system. 
		It is an application that monitors executions of repeated jobs such as bulding, testing a software project.
		It has 3 main features : 
		\begin{itemize}
			\item It allows a team to share common information easily.
			\item It executes automatically compilation, testing without human intervention.
			\item It keeps a track of previous productions and allow us to see their development. 
		\end{itemize}


		\includegraphics[scale=0.2]{Images/mylin.jpeg} 
		Mylin for eclipse connection. \\
		Mylin is the task and application lifecycle management framework for Eclipse
		It allows to visualize tasks from Redmine repositorie and it has a connector to Jenkins. \\ 
		It helps a developer to work efficiently with many different tasks (such as bugs, problem reports or new features). In a nutshell, it improves their productivity by reducing searching, scrolling, and navigation. \\

		\includegraphics[scale=0.1]{Images/ldap.jpeg} 
		Ldap authentification. \\
		Ldap is a protocol that allow us to access and maintain directory services. So we can access to some informations about the users of a network over TCP/IP protocol. 
		With the ldap authentification into redmine, the users don't need to create a Redmine's account but they can directly access into their redmine's project by using their ldap password and login.   
	\end{figure} 



\newpage

\subsection{Implementation}


I installed Redmine in ssh on an other machine. The installation was not very 
hard. I needed first to install Ruby and
Rails, and MySQL. I created an empty database and an user, then I set up the 
database connection configuration. 
To have an access to Redmine we need to run a server. But it's a bit annoying to start 
manually the server every time we restart the computer. 
That's why I searched how we can start a server automatically when we power up the computer. 
I add a daemon that starts the Redmine's server, and as long as the computer is on, 
we can directly access to our Redmine instance. 

Then I installed Jenkins plugin, meanwhile I learnt about SVN in order to add jobs into Redmine. 
SVN is a software versioning and revision control system, it allows developers to 
maintain current and historical versions of files such as source code, web pages, 
and documentation. 
 
The hardest part was to integrate Mylin into Eclipse. I needed a while to make it work but finally I just used the generic web connector of Eclipse. 

The integration of ldap authentification into Redmine was straightforward. I just 
needed to configure the authentification mode with ldap data.  

Once evrything installed, I just had to set up the environment in order to allow
the team's members to use it directly and easily.


\subsection{Project creation}

The last step was to create an instance for the project and look how to set up the environment in order to be nicest to use. 

I created a new project in which I add some members with their roles such as manager, 
developer or rapporteur. Then I looked at the issues service. In this service it is 
possible to add some issues such as bugs. It is probably the service that the developers
will most use. So I set up the environment by adding different types of issues such
as bugs, evolution, installation, development and assistance. 
I also add some new issue statuses for example. 
When we create a new issue, we first choose the type of the issue, then its status 
can be "New" or "To be covered". The issue is 
assigned to a project member who can later change the issue status from "New" to "Solved"
for example. I also add that it is mandatory to choose in which portlet the issue 
has been noticed. Issues can be nested within other issues, and also linked to each other.

 


I wrote some wiki pages related to the project. I add a jenkins page into the project
 in order to have an access to Jenkins directly from the project.  
 Once the Redmine's environment set up, every member of the project could add some
 issue, look at the issue assigned to them in order to fix them later. 



\cleardoublepage


