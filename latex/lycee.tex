\chapter{Rappels du lyc\'ee}
\section{Multiple et division euclidienne}
\begin{madef}
Soient a et b $\in \mathbb{Z} $
\newline
a est un multiple de b ssi $\exists k \in \mathbb{Z}$ tel que : \\
$a = kb $
\\ On dit aussi que  :
\begin{itemize}[label=\ding{217}, font=\color{black}]
	\item a est divisible par b 
	\item b est un diviseur a
	\item b divise a
\end{itemize}
\end{madef}


\begin{madef}
Soient a $\in \mathbb{Z}$ et b $\in \mathbb{N}$. \\
On appele \textbf{division euclidienne de a par b} l'op\'eration qui au couple (a,b) associe un couple (q,r) tel que : \\
$a = b \times q + r$ avec $0 \le r < b$ \\
\\
On appele a le dividende, b le diviseur, q le quotient et r le reste. 
\end{madef}


\section{Congruence}
\begin{madef}
Soient $n \in \mathbb{N}, n \ge 2$ et $a, b \in \mathbb{Z}$ 
On dit que deux entiers a et b sont congru modulo n ssi ils ont m\^eme restepar la division euclidienne par n. \\
On note alors : \\
$a \equiv b\ (\textrm{mod}\ n)$ ou $a \equiv b\ (n)$

\end{madef}

