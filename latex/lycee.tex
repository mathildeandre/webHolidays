\chapter{Rappels du lyc\'ee}
\section{Multiple et division euclidienne}
\begin{madef}
Soient a et b $\in \mathbb{Z} $
\newline
a est un multiple de b ssi $\exists k \in \mathbb{Z}$ tel que : \\
$a = kb $
\\ On dit aussi que  :
\begin{itemize}[label=\ding{217}, font=\color{black}]
	\item a est divisible par b 
	\item b est un diviseur a
	\item b divise a
\end{itemize}
\end{madef}


\begin{madef}
Soient a $\in \mathbb{Z}$ et b $\in \mathbb{N}$. \\
On appele \textbf{division euclidienne de a par b} l'op\'eration qui au couple (a,b) associe un couple (q,r) tel que : \\
$a = b \times q + r$ avec $0 \le r < b$ \\
\\
On appele a le dividende, b le diviseur, q le quotient et r le reste. 
\end{madef}

\begin{madef}
Soient a, b $\in \mathbb{N}$ \\
\textbf{pgcd : }\\
On appele pgcd(a,b) le plus grand commun diviseur de a et de b. \\
\textbf{ppcm : }\\
On appele ppcm(a,b) le plus petit commun multiple de a et de b. \\
\end{madef}

\begin{maprop}
Soient a, b $\in \mathbb{N}$ \\
$ ppcm(a,b) \times pgcd(a, b) = a \times b$
\end{maprop}

\begin{myproof}[Demonstration:]
Soient $m=ppcm(a,b)$ et $\delta = pgcd(a,b)$ \\
On a : $a|\delta$ et $b|\delta$ cad $\exists k, k' \in \mathbb{Z}$ tel
que $a=k \times \delta$ et $b=k' \times \delta $\\
On devrait alors avoir $m \times \delta = k \times \delta \times k' \times \delta$
$\Leftrightarrow m=k\times k'\times \delta$\\
Montrons donc que kk'$\delta = ppcm(a,b)$
\begin{itemize}[label=\ding{217}, font=\color{black}]
	\item $k k' \delta$ est un multiple de a et b cad a|m et b|m  ??\\
	On a $a = k \times \delta$ cad $k' \times a = k' \times k \times \delta$ cad $a | k'k\delta$ \\
	Idem pour b

	\item $kk'\delta$ est le \textbf{plus petit} multiple de a, b ??	
\end{itemize}
\end{myproof}

\begin{maprop}
Soient a, b $\in \mathbb{N}$ \\
$pgcd(a,b) = \delta \Leftrightarrow a\mathbb{Z}+b\mathbb{Z}=\delta\mathbb{Z}$
\end{maprop}

\begin{myproof}[D\'emonstration :]
Aide : $a\mathbb{Z}+b\mathbb{Z}=\{ak + bk' | k, k' \in \mathbb{Z}\}$
\begin{itemize}[label=\textbullet]
	\item $\Rightarrow$ Si $pgcd(a,b)= \delta$, montrons que 
	$a\mathbb{Z}+b\mathbb{Z}=\delta \mathbb{Z}$\\
	Soit $m\in a\mathbb{Z}+b\mathbb{Z}$ donc 
	$\exists a', b' \in \mathbb{Z}$ tel que $m=a\times a'+ b \times b'$ \\
	Or $\delta|a$ et $\delta|b$ donc $\exists k, k' \in \mathbb{Z}$
	tel que $a=k\times \delta$ et $b=k'\times \delta$ \\
	Donc $m=k\times \delta \times a' + k' \times \delta \times b'$
	$\Leftrightarrow m=\delta \times (ka'+k'b')$ cad $m \in \delta \mathbb{Z}$

	\item $\Leftarrow$ Si $a\mathbb{Z}+b\mathbb{Z}=\delta \mathbb{Z}$,
	montrons que $pgcd(a,b)=\delta$
	\begin{enumerate}
		\item Montrons que $\delta$ est un diviseur commun \'a a et b.\\
		$a = a \times 1 + b \times 0 \in a\mathbb{Z}+b\mathbb{Z}$ donc $a \in \delta \mathbb{Z}$ cad $\delta | a$ \\
		Idem pour b.
		\item Montrons que $\delta$ est bien le \textbf{plus grand} diviseur de a et b. \\
		Soit $\Delta$ un diviseur commun \'a a et b donc $\exists a', b' \in \mathbb{Z}$, $a=a'\Delta$ et $b=b'\Delta$\\
	 	Nous allons montrer que $\Delta | \delta$ cad $\Delta \le \delta$ \\
		$\delta \in \delta \mathbb{Z}$ donc $\delta \in a\mathbb{Z}+b\mathbb{Z}$ 
		donc $\exists k, k' \in \mathbb{Z}$ tel que  
		\begin{tabbing}
		\hspace{0.4cm} $\delta = ak + bk'$ \\
		$\Leftrightarrow \delta = a' \Delta k+b' \Delta k'$ \\
		$ \Leftrightarrow \delta = \Delta \times (ka'+k'b')$\\ 
		\end{tabbing}
		Donc $\Delta | \delta$ cad $\Delta \le \delta$ 
		cad $\delta = pgcd(a,b)$
	\end{enumerate}
\end{itemize}
\end{myproof}



\section{Nombres premiers}
\begin{madef}
Soit $n \in \mathbb{N}$. \\
On dit que n est un nombre premier s'il admet exactement deux diviseurs : 1 et lui-m\^eme.
\end{madef}

\begin{maprop}
Soit $n \in \mathbb{N} , n > 1$ 
\begin{enumerate}
	\item n admet au moins un diviseur premier
	\item si n n'est pas premier, n admet au moins un diviseur premier p tel que $p \le \sqrt{n}$ 
\end{enumerate}
\end{maprop}

\begin{myproof}[Demonstration:]
\begin{itemize}[label=\ding{182}, font=\color{black}]
	\item 
	\begin{itemize}[label=\ding{217}, font=\color{black}]
	\item Si n est premier, la propri\'et\'e est v\'erifi\'e : n|n
	\item Si n n'est pas premier, il admet dans $\mathbb{N}$ d'autres diviseurs que 1 et n. \\
	Soit p le plus petit diviseur de n. \\
	\textcolor{darkgreen}{p est-il premier?} \\
	Raisonnement par l'absurde : \\
	Si p n'est pas premier, alors appelons p' son plus petit diviseur.\\
	On a : $p'|p \Rightarrow p'|n $ mais $p'<p \Rightarrow $ \textbf{\textcolor{darkred}{Contradiction !}}  
	\end{itemize}
\end{itemize}
\begin{itemize}[label=\ding{183}, font=\color{black}]
	\item On a montr\'e que si n n'est pas premier il admet au moins un diviseur premier. Soit p ce diviseur. \\
	Alors p|n donc $\exists k \in \mathbb{Z}$ tel que $n=k \times p$. Donc k est aussi un diviseur de n et $k \ge$ p d'o\'u $n=pk \ge p^2$ donc $\sqrt{n} \ge p$.
\end{itemize}
\end{myproof}

\begin{matheo}
Il existe une infinit\'e de nombres premiers.
\end{matheo}

\begin{myproof}[Demonstration :]
Raisonnement par l'absurde: \\
Supposons que $\mathcal{P}$ est finit. Donc on peut \'ecrire $\mathcal{P} = \{p_1, p_2, p_3, \ldots , p_n\}$ \\
Consid\'erons k $\in \mathbb{N}$ tel que $k=p_1 \times p_2 \times \ldots \times p_n + 1 $ \\
$k \ge 2$, donc d'apr\`es la proposition pr\'ecedente, k poss\`ede un diviseur premier notons le q. \\
Le nombre q est l'un des $p_i$.\\
 Donc q|$p_1 \times p_2 \times \ldots \times p_n$ et q|k.\\
 Donc $q|k- p_1 \times p_2 \times \ldots \times p_n$. \\
 Donc $q|1$ cad q=1 mais 1 n'est pas premier $\Rightarrow$ Contradiction !!


\end{myproof}




\section{Congruence}
\begin{madef}
Soient $n \in \mathbb{N}, n \ge 2$ et $a, b \in \mathbb{Z}$ 
On dit que deux entiers a et b sont congru modulo n ssi ils ont m\^eme restepar la division euclidienne par n. \\
On note alors : \\
$a \equiv b\ (\textrm{mod}\ n)$ ou $a \equiv b\ (n)$
\end{madef}

\begin{matheo}
Soient $n \in \mathbb{N}, n \ge 2$ et $a, b \in \mathbb{Z}$ \\ 
$a \equiv b\ (\textrm{mod}\ n) \Leftrightarrow (a-b) \equiv 0\ (\textrm{mod}\ n)$

\end{matheo}
\begin{proof} $\Rightarrow$

\end{proof}


