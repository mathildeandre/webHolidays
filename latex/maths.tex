
\documentclass[12pt]{report}

\usepackage[latin1]{inputenc}    
\usepackage[T1]{fontenc}
\usepackage[francais]{babel} %langue française
\usepackage{layout} % Marges
\usepackage{color}
\usepackage{amssymb}
\usepackage{amsthm}
\usepackage{amsmath}
\usepackage{mathrsfs}
\usepackage{boiboites}
\usepackage{enumitem}
\usepackage{pifont}

\newtheoremstyle{break}%       %1 Nom
      {\topsep}%          %2 Espace avant
      	{\topsep}%          %2 Espace avant
	    {}%      %4 forme des caractères
	       {0pt}%          %5 indentation
	          { \sffamily \bfseries}   %6 Style de l'entête
		     {.}%             %7
		        {\newline}%      %8 Retour à la ligne après le titre
			   {}%             %9 Comme dans plain ?
\theoremstyle{break}%       %1 Nom





\newtheorem{maprop}{Proposition}[chapter]
%\newtheorem{madef}{D\'efinition}[chapter]
\newtheorem{monexo}{Exercice}[chapter]
\newtheorem{monex}{Exemple}[chapter]

\definecolor{darkgreen}{rgb}{0.0, 0.3, 0.0}
\definecolor{myyellow}{rgb}{1, 0.85, 0.00}
\definecolor{mygreen}{rgb}{0.15, 0.70, 0.1}



\newboxedtheorem[boxcolor=darkgreen, background=myyellow, titlebackground=mygreen!90,titleboxcolor = black, thcounter=chapter]{madef}{D\'efinition}{compteurTH}
\renewcommand{\thecompteurTH}{\arabic{chapter}.\arabic{compteurTH}}


\title{Cours de Math\'ematiques}
\author{Mathilde Andre}
\date{Vendredi 18 Juillet 2014}

\begin{document}
\maketitle
% Table des matieres
\renewcommand{\contentsname}{Sommaire}
\tableofcontents


\chapter{Rappels du lyc\'ee}
\section{Multiple et division euclidienne}
\begin{madef}
Soient a et b $\in \mathbb{Z} $
\newline
a est un multiple de b ssi $\exists k \in \mathbb{Z}$ tel que : \\
$a = kb $
\\ On dit aussi que  :
\begin{itemize}[label=\ding{217}, font=\color{black}]
	\item a est divisible par b 
	\item b est un diviseur a
	\item b divise a
\end{itemize}
\end{madef}


\begin{madef}
Soient a $\in \mathbb{Z}$ et b $\in \mathbb{N}$. \\
On appele \textbf{division euclidienne de a par b} l'op\'eration qui au couple (a,b) associe un couple (q,r) tel que : \\
$a = b \times q + r$ avec $0 \le r < b$ \\
\\
On appele a le dividende, b le diviseur, q le quotient et r le reste. 
\end{madef}

\begin{madef}
Soient a, b $\in \mathbb{N}$ \\
\textbf{pgcd : }\\
On appele pgcd(a,b) le plus grand commun diviseur de a et de b. \\
\textbf{ppcm : }\\
On appele ppcm(a,b) le plus petit commun multiple de a et de b. \\
\end{madef}

\begin{maprop}
Soient a, b $\in \mathbb{N}$ \\
$ ppcm(a,b) \times pgcd(a, b) = a \times b$
\end{maprop}

\begin{myproof}[Demonstration:]
Soient $m=ppcm(a,b)$ et $\delta = pgcd(a,b)$ \\
On a : $a|\delta$ et $b|\delta$ cad $\exists k, k' \in \mathbb{Z}$ tel
que $a=k \times \delta$ et $b=k' \times \delta $\\
On devrait alors avoir $m \times \delta = k \times \delta \times k' \times \delta$
$\Leftrightarrow m=k\times k'\times \delta$\\
Montrons donc que kk'$\delta = ppcm(a,b)$
\begin{itemize}[label=\ding{217}, font=\color{black}]
	\item $k k' \delta$ est un multiple de a et b cad a|m et b|m  ??\\
	On a $a = k \times \delta$ cad $k' \times a = k' \times k \times \delta$ cad $a | k'k\delta$ \\
	Idem pour b

	\item $kk'\delta$ est le \textbf{plus petit} multiple de a, b ??	
\end{itemize}
\end{myproof}

\begin{maprop}
Soient a, b $\in \mathbb{N}$ \\
$pgcd(a,b) = \delta \Leftrightarrow a\mathbb{Z}+b\mathbb{Z}=\delta\mathbb{Z}$
\end{maprop}

\begin{myproof}[D\'emonstration :]
Aide : $a\mathbb{Z}+b\mathbb{Z}=\{ak + bk' | k, k' \in \mathbb{Z}\}$
\begin{itemize}[label=\textbullet]
	\item $\Rightarrow$ Si $pgcd(a,b)= \delta$, montrons que 
	$a\mathbb{Z}+b\mathbb{Z}=\delta \mathbb{Z}$\\
	Soit $m\in a\mathbb{Z}+b\mathbb{Z}$ donc 
	$\exists a', b' \in \mathbb{Z}$ tel que $m=a\times a'+ b \times b'$ \\
	Or $\delta|a$ et $\delta|b$ donc $\exists k, k' \in \mathbb{Z}$
	tel que $a=k\times \delta$ et $b=k'\times \delta$ \\
	Donc $m=k\times \delta \times a' + k' \times \delta \times b'$
	$\Leftrightarrow m=\delta \times (ka'+k'b')$ cad $m \in \delta \mathbb{Z}$

	\item $\Leftarrow$ Si $a\mathbb{Z}+b\mathbb{Z}=\delta \mathbb{Z}$,
	montrons que $pgcd(a,b)=\delta$
	\begin{enumerate}
		\item Montrons que $\delta$ est un diviseur commun \'a a et b.\\
		$a = a \times 1 + b \times 0 \in a\mathbb{Z}+b\mathbb{Z}$ donc $a \in \delta \mathbb{Z}$ cad $\delta | a$ \\
		Idem pour b.
		\item Montrons que $\delta$ est bien le \textbf{plus grand} diviseur de a et b. \\
		Soit $\Delta$ un diviseur commun \'a a et b donc $\exists a', b' \in \mathbb{Z}$, $a=a'\Delta$ et $b=b'\Delta$\\
	 	Nous allons montrer que $\Delta | \delta$ cad $\Delta \le \delta$ \\
		$\delta \in \delta \mathbb{Z}$ donc $\delta \in a\mathbb{Z}+b\mathbb{Z}$ 
		donc $\exists k, k' \in \mathbb{Z}$ tel que  
		\begin{tabbing}
		\hspace{0.4cm} $\delta = ak + bk'$ \\
		$\Leftrightarrow \delta = a' \Delta k+b' \Delta k'$ \\
		$ \Leftrightarrow \delta = \Delta \times (ka'+k'b')$\\ 
		\end{tabbing}
		Donc $\Delta | \delta$ cad $\Delta \le \delta$ 
		cad $\delta = pgcd(a,b)$
	\end{enumerate}
\end{itemize}
\end{myproof}



\section{Nombres premiers}
\begin{madef}
Soit $n \in \mathbb{N}$. \\
On dit que n est un nombre premier s'il admet exactement deux diviseurs : 1 et lui-m\^eme.
\end{madef}

\begin{maprop}
Soit $n \in \mathbb{N} , n > 1$ 
\begin{enumerate}
	\item n admet au moins un diviseur premier
	\item si n n'est pas premier, n admet au moins un diviseur premier p tel que $p \le \sqrt{n}$ 
\end{enumerate}
\end{maprop}

\begin{myproof}[Demonstration:]
\begin{itemize}[label=\ding{182}, font=\color{black}]
	\item 
	\begin{itemize}[label=\ding{217}, font=\color{black}]
	\item Si n est premier, la propri\'et\'e est v\'erifi\'e : n|n
	\item Si n n'est pas premier, il admet dans $\mathbb{N}$ d'autres diviseurs que 1 et n. \\
	Soit p le plus petit diviseur de n. \\
	\textcolor{darkgreen}{p est-il premier?} \\
	Raisonnement par l'absurde : \\
	Si p n'est pas premier, alors appelons p' son plus petit diviseur.\\
	On a : $p'|p \Rightarrow p'|n $ mais $p'<p \Rightarrow $ \textbf{\textcolor{darkred}{Contradiction !}}  
	\end{itemize}
\end{itemize}
\begin{itemize}[label=\ding{183}, font=\color{black}]
	\item On a montr\'e que si n n'est pas premier il admet au moins un diviseur premier. Soit p ce diviseur. \\
	Alors p|n donc $\exists k \in \mathbb{Z}$ tel que $n=k \times p$. Donc k est aussi un diviseur de n et $k \ge$ p d'o\'u $n=pk \ge p^2$ donc $\sqrt{n} \ge p$.
\end{itemize}
\end{myproof}

\begin{matheo}
Il existe une infinit\'e de nombres premiers.
\end{matheo}

\begin{myproof}[Demonstration :]
Raisonnement par l'absurde: \\
Supposons que $\mathcal{P}$ est finit. Donc on peut \'ecrire $\mathcal{P} = \{p_1, p_2, p_3, \ldots , p_n\}$ \\
Consid\'erons k $\in \mathbb{N}$ tel que $k=p_1 \times p_2 \times \ldots \times p_n + 1 $ \\
$k \ge 2$, donc d'apr\`es la proposition pr\'ecedente, k poss\`ede un diviseur premier notons le q. \\
Le nombre q est l'un des $p_i$.\\
 Donc q|$p_1 \times p_2 \times \ldots \times p_n$ et q|k.\\
 Donc $q|k- p_1 \times p_2 \times \ldots \times p_n$. \\
 Donc $q|1$ cad q=1 mais 1 n'est pas premier $\Rightarrow$ Contradiction !!


\end{myproof}




\section{Congruence}
\begin{madef}
Soient $n \in \mathbb{N}, n \ge 2$ et $a, b \in \mathbb{Z}$ 
On dit que deux entiers a et b sont congru modulo n ssi ils ont m\^eme restepar la division euclidienne par n. \\
On note alors : \\
$a \equiv b\ (\textrm{mod}\ n)$ ou $a \equiv b\ (n)$
\end{madef}

\begin{matheo}
Soient $n \in \mathbb{N}, n \ge 2$ et $a, b \in \mathbb{Z}$ \\ 
$a \equiv b\ (\textrm{mod}\ n) \Leftrightarrow (a-b) \equiv 0\ (\textrm{mod}\ n)$

\end{matheo}
\begin{proof} $\Rightarrow$

\end{proof}


 



\chapter{Alg\`ebre}
Cours 1
\section{Quelques rappels sur $\mathbb{N}$}
\begin{maprop}
	Tout ensemble A non vide $\subset$ $\mathbb{N}$ a un plus petit \'el\'ement
\end{maprop}
\begin{madef}
	\textbf {Majorant} : On dit que M est un majorant de A $\subset$ $\mathbb{N}$ ssi $\forall$n $\in$ $\mathbb{N}$ n $\le$ M 
	\newline
	On dit aussi que A est major\'e
\end{madef}
\begin{madef}
	\textbf {Relation d'\'equivalence} : Soit $\mathcal{R}$ une relation binaire sur A $\subset$ $\mathbb{N}$. \newline
	$\mathcal{R}$ est une relation d'\'equivalence ssi elle est:
	\begin{enumerate}
		\item reflexive :  $\forall$ x $\in$ A, x$\mathcal{R}$x 
		\item symetrique : $\forall$ (a,b) $\in$ A$^2$, si a$\mathcal{R}$b $\Rightarrow$ b$\mathcal{R}$a  
		\item transitive : $\forall$ (a,b,c) $\in$ A$^3$, si a$\mathcal{R}$b et b$\mathcal{R}$c $\Rightarrow a\mathcal{R}c$
	\end{enumerate}

	\textbf {Classe d'\'equivalence} : La classe d'\'equivalence de x pour $\mathcal{R}$ est tous les y tel que x$\mathcal{R}$y, on la note $\overline{x}$

\end{madef}




\section{Construction de $\mathbb{Z}$} 
\textcolor{darkgreen}{Comment construire $\mathbb{Z}$ ? }

Soit $\mathcal{R}$ une relation d'\'equivalence sur $\mathbb{N} \times \mathbb{N}$  d\'efinit ainsi : \newline
$\forall (a,b) \in A^2$ et $(a',b') \in A^2$ , $(a,b)\mathcal{R}(a',b')$ ssi $ a+b'=a'+b $

\textcolor{darkgreen}{Quelles sont les classes d'\'equivalences de (0, 0) et (0, a) ?}

\begin{enumerate}
	\item $\overline{(0, 0)}$ = $\{(x,y) \in \mathbb{N} \times \mathbb{N}$, $(x,y)\mathcal{R}(0,0)\}$ = $\{(x,y) \in \mathbb{N} \times \mathbb{N}$, $x=y \}$ = $\{(x,x)$, $x \in \mathbb{N} \}$
	\item $\overline{(0, a)}$ = $\{(x,y) \in \mathbb{N} \times \mathbb{N}$, $x+a=y\}$ = $\{(x, x+a)$, $x \in \mathbb{N}\}$
\end{enumerate}

On a : $\overline{(a,b)} + \overline{(c, d)}$ = $\overline{(a+c, b+d)}$
\newline
On a donc : $\overline{(0, a)} + \overline{(a,0)}$ = $\overline{(a,a)}$ = $\overline{(0,0)}$
\newline
Et on note : $\overline{(a, 0)} = -a$ 
\newline

\textbf {La d\'emonstration par r\'ecurrence : }
\newline
On va montrer que P(n) vraie pour tout n $\in \mathbb{N} \Leftrightarrow $ 
\begin{enumerate}
	\item P(0) vrai
	\item Supposons P(n) vrai alors P(n+1) vrai
\end{enumerate}
Supposons $\mathcal{P}$(0) vrai et \newline
Si $\mathcal{P}$(n) vrai $\Rightarrow \mathcal{P}$(n+1) vrai \newline
On va faire une d\'emonstration par l'absurde : \newline
Il existe un m $\in \mathbb{N}$, $\mathcal{P}(m)$ faux \newline
Soit $A = \{n \in \mathbb{N}, \mathcal{P}(n) faux \}$ 
\newline
$A \subset \mathbb{N} \Rightarrow$ A admet un plus petit element, appelons le i.  
\newline
Donc $i \neq 0$ et $\mathcal{P}(i-1)$ est vrai. \newline
D'apr\`es notre supposition on a alors $\mathcal{P}(i)$ vrai : CONTRADICTION

\section{Les groupes}
\begin{madef}
	On dit que $(G, \ast)$ est un groupe avec G un ensemble et $\ast$ une loi sur G ssi :
	\begin{enumerate}
		\item $\ast$ est associative cad $\forall x, y, z \in G$ $(x\ast y)\ast z = x*(y*z)$
		\item G admet un \'element neutre : $\exists e \in G$, $\forall x \in G$, $x*e=e*x=x $ 
		\item Tout \'element de G admet un sym\'etrique : \newline $\forall x \in G$, $\exists x^{-1}$, $x*x^{-1}=x^{-1}*x=e$
	\end{enumerate}
	On dit qu'un groupe est ab\'elien ou commutatif si $\ast$ est commutative. 
\end{madef}
\begin{monex}
	Exemple de groupe non ab\'elien : Les permutations \newline
	$a=(\begin{smallmatrix}
		1 & 2 & 3\\
		1 & 2 & 3 
	\end{smallmatrix})$
	$b=(\begin{smallmatrix}
		1 & 2 & 3\\
		1 & 3 & 2 
	\end{smallmatrix})$
	$c=(\begin{smallmatrix}
		1 & 2 & 3\\
		2 & 1 & 3 
	\end{smallmatrix})$
	\newline
	\\
	Calculer $a\circ b $ puis $b \circ a$ 
	\newline
	\\
	$b\circ c = (\begin{smallmatrix}
		1 & 2 & 3\\
		3 & 1 & 2 
	\end{smallmatrix})$
	\newline
	$c\circ b = (\begin{smallmatrix}
		1 & 2 & 3\\
		2 & 3 & 1 
	\end{smallmatrix})$
	\newline
	Donc l'ensembre des permutations muni de la loi de composition n'est pas un groupe ab\'elien.
\end{monex}

\subsection{Les sous groupes}
\begin{madef}
	On dit que $(H, \ast) \subset G$ un ensemble et $\ast$ est un sous-groupe de G ssi :
	\begin{enumerate}
		\item $H \neq \emptyset$
		\item H admet le m\^eme \'el\'ement neutre que G
		\item H est stable : $\forall x, y \in G, x*y \in H$ 
	\end{enumerate}
\end{madef}
\begin{monex}
	\textcolor{darkgreen}{Quels sont les sous-groupes de $\mathbb{Z}$ ?}
	\newline
	Les sous groupes de $\mathbb{Z}$ sont les $k\mathbb{Z}$
	\newline
	$k\mathbb{Z} = \{\forall x \in Z, kx\}$ 
	\newline
	\\
	\textbf {Demo :}
	Soit H un sous groupe de $\mathbb{Z}$ ne contenant pas 0
	\newline
	$H \cap \mathbb{N^*} \in \mathbb{N}$ est non vide donc il admet un plus \'el\'ement, notons le k \newline
	Soit $h \in H \cap \mathbb{N^*}$ alors division euclidienne de h par k : $\exists (q,r) \in \mathbb{Z} \times \mathbb{H}$ tel que h = k*q+r ac $0 \le r < k$ mais k est le plus petit \'el\'ement de H donc r=0. 
\end{monex}

\subsection{Morphisme de groupe}
\begin{madef}
	Soient $(G_1, \ast_1) et (G_2, \ast_2)$ deux groupes, et
	$\phi : G_1 \longrightarrow G_2$, 
	\newline 
	$\phi$ est un morphisme de groupe ssi:
	$\phi(x_1*_1x_2) = \phi(x_1) *_2 \phi(x_2)$ avec $x_1, x_2 \in G_1$
\end{madef}

\subsection{Noyau}
\begin{madef}
	Soient $(G_1, \ast_1) et (G_2, \ast_2)$ deux groupes, et
	$\phi : G_1 \longrightarrow G_2$, 
	\newline
	On note $Ker(\phi) = \{y\in G_1, \phi(y)=e_2\} $
\end{madef}

\begin{maprop}
	$Ker(\phi)=\{\emptyset\} \Leftrightarrow \phi $ est injective 
	\newline
	 
	\begin{itemize}\renewcommand{\labelitemi}{$\bullet$}
		\item Si $\phi$ injective alors si $ x, y \in G_1$ et $\phi(x) = \phi(y) \Rightarrow x=y $
		\begin{tabbing}
	 	\hspace{0.5cm}$\phi(x) = \phi(y)$ \\ 
	 	$ \Leftrightarrow \phi(x) * \phi(y)^{-1} = e_2 $ \\
		$\Leftrightarrow \phi(x) * \phi(y^{-1}) = e_2 $ \\
	 	$\Leftrightarrow \phi(x*y^{-1})=e_2  $ \\ 
		Or $x=y $\\
		\hspace{0.4cm} $x*y^{-1}=e_1$\\
		\end{tabbing}
		Donc $\phi(e_1) = e_2 $ et $Ker(\phi) = \{\emptyset\}$ \\
	
		\item Si $Ker(\phi)=\{\emptyset\}$ :\\
		 Soient $x, y \in G_1 $ tel que $\phi(x)=\phi(y)$.
		 \begin{tabbing}
	 	Alors
	 	$\phi(x) * \phi(y)^{-1} = e_2 $ \\ 
	 	\hspace{0.4cm} $ \Leftrightarrow \phi(x) * \phi(y^{-1}) = e_2 $ \\
		\hspace{0.4cm} $\Leftrightarrow \phi(x*y^{-1}) = e_2 $ \\
	 	\hspace{0.4cm} $\Leftrightarrow x*y^{-1}=e_2  $ \\ 
	 	\hspace{0.4cm} $\Leftrightarrow x = y  $
		\end{tabbing}
		Donc $\phi$ est injective.
	\end{itemize}
\end{maprop}

\subsection{Groupe quotient}

\newpage
kzkzk





\end{document}
