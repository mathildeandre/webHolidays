
\subsection{Implementation}


I installed Redmine in ssh on an other machine. The installation was not very hard but allow me to know better the IT infrastructure. 
To have an access to Redmine we neeed to run a server. But it's a bit annoying to have to start the server every day or even more. 
That's why I searched how we can start a server automatically when we power up the computer. 
I add a daemon for the Redmine's server, and we can access to our Redmine instance directly. \\ 

Then I installed Jenkins plugin, meanwhile I learnt about SVN in order to add jobs into Redmine. 
SVN is a software versioning and revision control system, it allows Developers to maintain current and historical versions of files such as source code, web pages, and documentation. \\
 
The hardest part was to integrate Mylin into Eclipse. I needed a while to make it work but finally I just used the generic web connector of Eclipse. 

The integration of ldap authentification into Redmine was straightforward. I just needed to configure the authentification mode with ldap data. \\ 

In a nutshell : Schema

\subsection{Project creation}

The last step was to create an instance for the project and look how to set up the environment in order to be nicest to use. 

I wrote some wiki pages related to the project. I add a jenkins page into the project in order to have an access to Jenkins directly from their project.  
Finally I looked how to create new issues in order to be ready to report bugs into Redmine. 


