

\newpage
Présenter les objectifs du stage afin que vos évaluateurs cernent bien la limite entre l’existant et votre contribution réelle : (environ 5 pages)
\begin{itemize}
	\item le travail à réaliser ou le problème à résoudre
	\item l’état de l’art des solutions existantes et des contraintes fixées par l’entreprise
	\item votre solution motivée à partir de l’analyse ci dessus
\end{itemize}


---------------------------------------------------------------

\newpage
\section{Goals}
My tasks were related to differents fields such as installing new software for the team 
of developers, finding bugs and evolution, reporting them into an application 
and finally fixing some of them. 

In this part I will present these tasks separatly. For each of them, I will develop 
the demand of the company, the limit they fix, and finally present my solution. 

\subsection{Redmine installation}
As I said before, they are a team of 4 developers in the company. 
They all work with an IDE for j2EE development : Eclipse. 
They also use SVN, a software versionning and source code management 
system. They now needed a tool for project management. 
They wanted a software that would allow them to report bugs in a easy way and keep a 
track of their work. Indeed, they were in a stage where they had to test the application
over and over. So they would need to write about bugs they found, assigned them to
somebody in order to be solved. In a nutshell, they needed to keep track of the bugs. 
After some research, they found that Redmine application was a good project management
tool, that's why I had to install this software. 

Redmine is a tool for project management and bug-tracking. It is quite easy to begin
with the software, even for a non technical person. Moreover, it is very scalable 
thanks to the plugins we can install with it. 

The developers wanted to have an access into Redmine directly from their IDE (Eclipse), 
in order to know very easily what bug they had to fix or change the statue of an issue.

Redmine software has several useful plugins available. My task were to find some plugins that fit the 
developers demands I described before. 
Redmine is an open source software and is easy to install and use. 
First I installed Redmine into a web page shared by the company.  
Then I did some research about all the plugins in order to choose the most useful, 
the goal was to make developers work faster. 
At last, I set up Redmine's environment to allow developers to use it as easily as possible. 





\newpage
\subsection{System testing}
The company had some code that they didn't develop by themselves that's why they didn't 
know if the application was working exactly as expected but it was almost functionnal. 
In order to be sure it has as few bugs as possible, they were in a stage where they 
have to test the application as deeply as possible.  
I will give you an example of problems a lot of developers encountered : it is the 
compatibility of different browsers. When you develop a web application, you always have to
install as many browsers as possible in order to test your application on them all.
On the Pentila application, some services were very well designed and totally functionnal
 on Firefox but on Internet Explorer, the design was awfull, we couldn't read all the 
 sentences of a post-it for example or some pictures weren't displayed. 


The company had one document with an acceptance test plan for some services of the 
application. This document was not finished at all but it was very useful in order 
to continue the testing stage later. 
My task was first to finish this document, I had to write all the functionalities of every
services into the document. Once done, I would have to test all the services 
of the website, and report what I found into the Redmine application. 

In order to accomplish this task, first of all I did some acceptance testing to continue
 and finish to write the document. Then I did some regression testing. After each new 
 version of the application, I test the services by following the steps of the acceptance 
 test plan. 

Finally, I looked at the existing softwares for functionnal tests in order to do automated 
tests. One could be very useful for the company, it's called JMeter. The other intern 
installed it and register some scenarios into it. At the end, I helped him to install
a tool useful to use with JMeter sorftware, it is the jconsole that we can use with Tomcat
server. I will describe a bit how I did it later in this report. 



\newpage
\subsection{Bugs fixing}
Once the bugs tracking done, the next stage was to fix the bugs reported. 
The bugs were related to different areas of the application
such as Tomcat, ldap or programming bugs. Most of them was CSS, Html, Javascript or j2EE
bugs. Here is an example of bug I had to solve. On the Articles service, we could see
all the articles posted by your group or by the headteacher, but the user could also
write a new article for a group and add a media into it. 
When we clicked on the add Media button, 
the user could choose a file from his local computer or a file from his pigeonhole or school
bag services. 
The file has to be with a good extension. But here is the problem : some names of files
from the pigeonhole and schoolbag was displayed twice in the list.  

I was asked to fix some of the bugs such as the one described above. 
In order to do that, I needed to have an access to the source code of the application
 and make it work on my computer.
The company gave me a document that explain how to install the source code on OS Debian. 
I had to follow the steps described in it, in order to have access to all their codes 
and their classes.

The first step was to understand how the application worked, and then install the working 
environment on my computer. 
So I made some researches about Tomcat, Liferay, Solr and ldap and then installed everything. 
Once the environment working, I started to correct some bugs. I quickly understood that
the bugs assigned to me would be related to different areas.  
That's why I followed some tutorials about all of these web programming languages. I 
needed to know at least how was CSS and Javascript working. I already know quite well
Html.  

In order to correct the bugs assigned to me, I first had to reproduce them, then I had
to look at the code source to find in which class the bug was related. I looked at 
the navigator debugger tool of Firefox for example. This tool prooves to be very helpful 
to figure out what was the cause of the bug. I will describe in details
the different steps to correct bugs of the application later in the   
\hyperlink{ancre}{Bugfix} part of this report.
